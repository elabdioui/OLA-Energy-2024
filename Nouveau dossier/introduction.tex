

\documentclass[a4paper,12pt]{report} % Assurez-vous d'avoir la déclaration de document class

\usepackage{fancyhdr} % Pour la gestion des en-têtes et pieds de page
\usepackage{titlesec} % Pour le contrôle des titres

\markboth{\MakeUppercase{Introduction}}{}%
\addcontentsline{toc}{chapter}{Introduction}%

\begin{document}

\chapter*{Introduction} % Ajout du chapitre ici, car vous utilisez \addcontentsline

Dans le cadre de notre stage chez OLA Energy et en collaboration avec la première édition de l'EMSI IT Summer Competition 2024, nous avons entrepris le développement d'un module de gestion de production et de gestion des stocks assisté par ordinateur (GMAO) pour le projet "LPG IN THE BOX". Cette compétition innovante, organisée par l'École Marocaine des Sciences de l'Ingénieur (EMSI), nous a offert une plateforme unique pour mettre en pratique nos compétences techniques et répondre à des problématiques réelles de l'industrie, en partenariat avec des entreprises de premier plan telles qu'OLA Energy.

Le projet "LPG IN THE BOX" a pour objectif de centraliser l’ensemble des flux d’information pour améliorer la gouvernance de l’activité Gaz dans les filiales du groupe à travers l'Afrique. Il s’agit d’un projet ambitieux qui inclut l’analyse et l'amélioration des outils existants de gestion de production et de stocks, tout en intégrant des technologies modernes telles que VBA et Power Apps pour une meilleure visualisation et un suivi optimisé des opérations. En tant que l'un des principaux acteurs du secteur énergétique africain, OLA Energy vise à renforcer sa capacité de gestion stratégique en optimisant ses processus et en améliorant l'efficacité de ses systèmes d’information.

Le cadre de l'EMSI IT Summer Competition 2024 nous a permis de travailler sur un projet réel tout en bénéficiant d’un encadrement de haut niveau. Grâce au soutien et aux conseils de **Charafeddine LECHHEB** et **Hamza YOUBI**, nos encadrants professionnels au sein d'OLA Energy, ainsi qu'à l'accompagnement académique de **Mme Chourouk ELOKRI**, nous avons pu aborder ce projet de manière méthodique et rigoureuse. L'accent a été mis sur l'analyse des besoins des utilisateurs, la conception de solutions adaptées, le développement de nouvelles fonctionnalités, et la validation par des tests approfondis, assurant ainsi que chaque amélioration apportée réponde aux exigences spécifiques du projet et aux standards de qualité d'OLA Energy.

Cette compétition, en tant qu'initiative pionnière, nous a également permis d'explorer les nouvelles tendances en matière de gestion des systèmes d'information et d'infrastructure logicielle. Les solutions proposées, incluant des sous-modules pour le suivi des achats et des consignations, ainsi que la création d’une bibliothèque technique dédiée à l’activité GPL, représentent des avancées significatives dans l'amélioration de la gestion des ressources. Ce projet nous a offert une expérience enrichissante, marquant une étape importante dans notre parcours professionnel et notre contribution au développement technologique du continent africain.

En participant à l'EMSI IT Summer Competition 2024, nous avons non seulement eu l'occasion de mettre en pratique nos compétences théoriques, mais aussi de contribuer à un projet d'envergure qui a un impact direct sur la performance opérationnelle et la compétitivité d'OLA Energy en tant que leader du secteur énergétique africain.

\end{document}
